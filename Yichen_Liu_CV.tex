\documentclass[10pt]{article} % minimum 10pt
\usepackage[slantfont, boldfont]{xeCJK}
\usepackage{amsmath}
\usepackage{enumitem}
\usepackage[left=0.4in,top=0.4in,right=0.4in,bottom=0.8in, letterpaper]{geometry}
\usepackage{hyperref}
\usepackage[x11names]{xcolor}
\hypersetup{
    colorlinks=true,
    linkcolor=blue,
    filecolor=blue,
    citecolor=black,      
    urlcolor=Blue3, % url color
    }
\linespread{1.15}%行距 1
% \setlength{\parskip}{1.5ex}%段距 1.5ex
\setlength{\parindent}{0em}%缩进 0em
\usepackage{fancyhdr}
\pagestyle{fancy}
\fancyhf{}
\renewcommand{\footrulewidth}{0.4pt}
\lfoot{Yichen Liu}
\cfoot{\today}
\rfoot{\thepage\hspace{1pt} OF \pageref*{LastPage}}
\renewcommand\headrulewidth{0pt}
% \setlength{\headheight}{0in}%固定页眉位置 页眉与正文baseline的高度差 加上之后其余页就同步了
% \setlength{\footskip}{0in}%固定正文结尾 结尾下限和页脚下限的高度差 加上之后其余页就同步了
% \setlength{\headsep}{0in}
\usepackage{titlesec}
\titleformat{\section}{\vspace{-1.5ex}\bf}{}{0em}{}[\hrule height 0.8pt\vspace{-1.5ex}]
\usepackage{lastpage}
\usepackage{fontspec}
% \setmainfont{Garamond}
\setmainfont{Cambria}
% \setmainfont{Georgia}
% \setmainfont{Book Antiqua}
\usepackage{enumitem}
\setenumerate[1]{itemsep=0pt,partopsep=0pt,parsep=\parskip,topsep=5pt}
\setitemize[1]{itemsep=0ex,partopsep=0ex,parsep=0ex,topsep=0ex}
\setitemize[2]{itemsep=0ex,partopsep=0ex,parsep=0ex,topsep=0ex}
% \setdescription{itemsep=0pt,partopsep=0pt,parsep=\parskip,topsep=5pt}
\usepackage{academicons}
\newcommand{\orcid}[1]{\href{https://orcid.org/#1}{\textcolor[HTML]{A6CE39}{\aiOrcid}}}
\newcommand{\googlescholar}[1]{\href{https://scholar.google.com.hk/citations?user=#1}{\textcolor[HTML]{3983FE}{\aiGoogleScholar}}}
\usepackage{fontawesome}
\newcommand{\github}[1]{\href{https://github.com/#1}{\textcolor[HTML]{000000}{\faGithub}}}
\usepackage{booktabs}% http://ctan.org/pkg/booktabs
\newcommand{\tabitem}{~~\llap{\textbullet}~~}
\usepackage{multicol}


\begin{document}

\begin{center}\bf{\large{YICHEN LIU}}\end{center}

\begin{multicols}{2}

\begin{tabular}{rl}
    tel & +1-(447) 902-2638 \\
    e-mail & \href{mailto:yl127@illinois.edu}{yl127@illinois.edu} \\
    web & \href{https://yliu.fit}{https:/\!/yliu.fit}
\end{tabular}

\begin{tabular}{rl}
    \orcid{0000-0003-4247-0169} & \href{https://orcid.org/0000-0003-4247-0169}{0000-0003-4247-0169} \\
    \github{Chisen-Lupus} & \href{https://github.com/Chisen-Lupus}{Chisen-Lupus} \\
    \googlescholar{GRjhRLUAAAAJ} & \href{https://scholar.google.com.hk/citations?user=GRjhRLUAAAAJ}{Yichen Liu} \\
\end{tabular}

\end{multicols}

\begin{section}{EDUCATION}

\textbf{University of Illinois at Urbana-Champaign} | College of Liberal Arts \& Sciences \hfill \textbf{Aug 2022 - Present}
\begin{itemize}[leftmargin=1.5em]
    \item Bachelor of Science (\textbf{Honor}) in \textbf{Astrophysics} and \textbf{Mathematics}, expected May 2024 \hfill \textbf{4.00/4.00} 
    \item Minor in \textbf{Physics}, \textbf{Computer Science}, and \textbf{Chemistry}
    % \item Enrolling in the \textbf{LAS Honors program} 
\end{itemize}
\textbf{University of Macau} | Faculty of Science and Technology \hfill \textbf{Aug 2019 - May 2022}
\begin{itemize}[leftmargin=1.5em]
    \item Completed Junior Year of \textbf{Applied Physics and Chemistry (Honour)}  %\hfill \textbf{3.40/4.00} (last year: 3.81)
    % \item Enrolled in the \textbf{Honours College}
    % \item Director of \textbf{University of Macau Physics Society} \hfill Aug 2020 - Feb 2021
\end{itemize}
    
\end{section}

\begin{section}{PUBLICATIONS}
    
\begin{enumerate}[leftmargin=1.5em]
    % \item Yichen Liu, et al., DES dwarf AGNs, In prep.
    \item Colin J. Burke, Yichen Liu, Charlotte A. Ward, Xin Liu, Jenny Greene, Priya Natarajan, ``Host galaxy properties of HSC-SSP variable AGNs in the COSMOS field and expectations for Rubin Observatory'', In prep.
    \item Grant Merz, \textbf{Yichen Liu}, Colin J. Burke, Patrick D. Aleo, Xin Liu, Matias Carrasco Kind, Volodymyr Kindratenko, Yufeng Liu, \href{https://academic.oup.com/mnras/advance-article-abstract/doi/10.1093/mnras/stad2785/7273850?redirectedFrom=fulltext}{``Detection, Instance Segmentation, and Classification for Astronomical Surveys with Deep Learning (DeepDISC): Detectron2 Implementation and Demonstration with Hyper Suprime-Cam Data,"} MNRAS 526, 1122 (2023)
    \item \textbf{Yichen Liu}, Peixia Zheng, and Hong-Chao Liu, \href{https://opg.optica.org/oe/fulltext.cfm?uri=oe-30-9-14073&id=471300}{``Anti-loss-compression image encryption based on computational ghost imaging using discrete cosine transform and orthogonal patterns,"} Opt. Express 30, 14073 (2022)
    \item Peixia Zheng, \textbf{Yichen Liu}, and Hong-Chao Liu, \href{http://www.irla.cn/cn/article/doi/10.3788/IRLA20211058}{``Single-pixel imaging and metasurface imaging,"} Infrared and Laser Engineering 50, 20211058-1 (2022)
\end{enumerate}

\end{section}

\begin{section}{RSEARCH EXPERIENCES}

\textbf{Research Assistant} at \textit{Department of Astronomy, University of Illinois} \hfill \textbf{Sep 2022 - Present} 
\begin{itemize}[leftmargin=1.5em]
    \item Advisor - \href{mailto:xinliuxl@illinois.edu}{Professor Xin Liu}
    \item Project 1 - instance segmentation in astronomical surveys using machine learning (NCSA SPIN internship) %\hfill Sep 2022 - Present
    \begin{itemize}[leftmargin=1.5em]
        \item Examined the source extraction pipelines of \href{https://github.com/burke86/deepdisc}{\texttt{DeepDISC}} and \href{https://github.com/burke86/astro_rcnn}{\texttt{Astro R-CNN}} using \href{https://github.com/kbarbary/sep/tree/v1.1.x}{\texttt{Sep}} and \href{https://github.com/pmelchior/scarlet}{\texttt{Scarlet}}
        \item Conducted simulation runs of different models and configurations based on PhoSim data on Hardware - Accelerated Learning (HAL) cluster using \href{https://github.com/facebookresearch/detectron2}{\texttt{Detectron2}}
        \item Modified code and applied Transformer models, MViT and VitDet, into the pipeline
        \item Recent work includes building neural networks for photometric redshift estimation 
    \end{itemize}
    \item Project 2 - DES SED fitting %\hfill Feb 2023 - May 2023
    \begin{itemize}[leftmargin=1.5em]
        \item Performed SED fitting of the sources in the DES and WISE catalogs using \href{https://cigale.lam.fr/}{\texttt{CIGALE}} 
        \item Generated criteria and selected AGN candidates in the source catalogs
    \end{itemize}
    \item Project 3 - host galaxy properties of variable AGNs %\hfill Jun 2023 - Present
    \begin{itemize}[leftmargin=1.5em]
        \item Matched the dwarf AGN candidates in HSC DR2 catalog to DR3, SIMBAD, and COSMOS2020 databases 
        \item Prepared batch-download code of optical spectra for SDSS, zCOSMOS, Magellan, DEIMOS, etc. 
        \item Compared and resolved the inconsistensies of redshifts between HSC and other databases
        \item Performed SED fitting on the candidates and concluded the relation between black hole masses and redshifts
    \end{itemize}
    \item Project 4 - redshift estimation in astronomical surveys using machine learning (NCSA SPIN internship)
\end{itemize}

\textbf{Summer Research Internship} at \textit{National Observatory of China} \hfill \textbf{Jun 2022 - Aug 2022} 
\begin{itemize}[leftmargin=1.5em]
    \item Advisor - \href{mailto:chjwu@bao.ac.cn}{Professor Chaojian Wu}
    \item Project - meteor slitless spectrum
    \begin{itemize}[leftmargin=1.5em]
        \item Generated the spectrum of 2021 Gemini meteors captured by DSLR
        \item Analyzed the intensitities of Sodium and Magnesium lines using Python
        \item Wrote machine learning code to filter and locate meteors from mass recording and perform photometry automatically
    \end{itemize}
\end{itemize}

\textbf{Research Assistant} at \textit{Institute of Applied Physics \& Materials Engineering, University of Macau}\hfill \textbf{Aug 2019 - May 2022} 
\begin{itemize}[leftmargin=1.5em]
    \item Advisor - \href{mailto:hcliu@um.edu.mo}{Professor Hongchao Liu}
    \item Project 1 - ghost imaging in complex environment %\hfill Aug 2019 - Aug 2021
    \begin{itemize}[leftmargin=1.5em]
        \item Reviewed latest studies on ghost imaging \& single-pixel imaging and presented research summaries at staff meetings
        \item Measure ghost imaging quality based on different equipment and reconstruction algorithms in MATLAB, analyzed the data, and authored reports for project supervisors
        \item Investigate light patterns reflected by distorting mirrors and compare the patterns to reflections from regular mirrors
    \end{itemize}
    \item Project 2 - anti-loss image encryption based on ghost imaging %\hfill Sep 2021 - Feb 2022
    \begin{itemize}[leftmargin=1.5em]
        \item Conducted experiments on ghost imaging, metamaterials and metasurfaces, and topological materials
        \item Designed Python algorithms based on compressive sensing and gradient descent
        \item Managed computational imaging simulations using \href{https://pytorch.org/}{\texttt{PyTorch}} using high-performance graphic card
        \item Published a high-impact article as the first author \href{https://www.tdm.com.mo/en/news-detail/683438?isvideo=false&lang=en&category=all}{as the first undergraduate student in the department} 
    \end{itemize}
    \item Project 3 - ghost imaging using recurrent neural network %\hfill Mar 2022 - May 2022
    \begin{itemize}[leftmargin=1.5em]
        \item Operated laser devices in collaboration with postgraduate students
        \item Summarized and verified existing ghost imaging methods that involve nerural networks
        \item Designed Python pipelines based on recurrent and convolutional neural network for ghost imaging 
    \end{itemize}
\end{itemize}

\end{section}

\begin{section}{OBSERVATION EXPERIENCE}
    
\begin{itemize}[leftmargin=1.5em]
    \item Cerro Tololo Inter-American Observatory, Blanco 4m / DECam: 3 nights observation \hfill Jan 2023 - Apr 2023
\end{itemize}

\end{section}

\begin{section}{AWARDS AND GRANTS}

\begin{itemize}[leftmargin=1.5em]
    \item AAS 243rd Meeting Travel Grants, Department of Astronomy \hfill Oct 2023
    \item NCSA SPIN Internship (Summer 2023 \& Academic Year 23-24) \hfill Aug 2023 
    \item University of Illinois Dean's Honor List (2022-2023) \hfill Jul 2023
    \item Smart Star Sponsorship for studies at University of California, Berkeley \hfill Jun 2022
    \item University of Macau Dean's Honour List (2020 and 2022) \hfill Aug 2022
    \item Residential College Summer Programme Sponsorship for studies at Shanghai Jiao Tong university \hfill May 2021
    \item Third Prize, China Undergraduate Physics Tournament \hfill Oct 2020
    \item National Team Leader at the 2019 European Union Contest for Young Scientists \hfill Sep 2019
    \item University of Macau Full Scholarship (2019-2021) \hfill Aug 2019
    \item Bronze Medal, International Olympiad of Astronomy and Astrophysics \hfill Nov 2018
    \item First Prize, China Adolescents' Science and Technology Innovation Contest \hfill Aug 2018
    \item Second Prize, Deng Feng National Contest on Science and Innovation \hfill Aug 2018
    \item Second Prize, China National Astronomy Olympiad \hfill May 2018
\end{itemize}
    
\end{section}

\begin{section}{TEACHING}

\textbf{Undergraduate Tutor} \hfill \textbf{Jan 2023 - May 2023}
\begin{itemize}[leftmargin=1.5em]
    \item Planned and facilitated collaborative tutoring sessions for Astronomy-program major students enrolled in introductory-level thermal physics, quantum physics, and astrophysics
\end{itemize}

\textbf{Youtuber in physics and mathematics} \hfill \textbf{Sep 2021 - Present}
\begin{itemize}[leftmargin=1.5em]
    \item Live stream or publish videos in ``Bilibili'' platform, offering public education resources in Chinese
    \item Topics include undergraduate-level or self-leared mathematics and physics, such as complex variables
    \item \href{https://www.bilibili.com/video/BV1th411W7xu/}{The most popular video} obtained more than 160,000 watchings
\end{itemize}

\textbf{Seminar of Physics at the University of Macau} \hfill \textbf{Feb 2022 - May 2022}
\begin{itemize}[leftmargin=1.5em]
    \item This was a series of unofficial lectures organized by me and my classmate, \href{http://runawayfancy.me/}{Jiheng Duan}, offering math and physics contents that the University of Macau's curriculum did not provide, such as classical mechanics and partial differential equations, supplementing the theoretical basis of future research and studies in physics for DPC students 
    \item Conducted the class meetings twice a week over the semester
    \item Prepared and taught the lecture \textit{SPUM 102 The tools of physical tools}, including complex variables, $\Gamma$ functions, integral transforms, $\delta$ functions, and Green functions \hfill {\footnotesize \href{https://github.com/Chisen-Lupus/Seminar-of-Physics-UM/blob/main/SPUM%20102%20The%20tools%20of%20physical%20tool.pdf}{|Syllabus}}
    \item The recordings of the lectures are publically available on \href{https://www.youtube.com/watch?v=nQkv03r-XeQ&list=PLV9fHDZW7hHWQ9rrAk7c9kdeV-Lqyt7pV&index=10}{Youtube}
    % \item Wrote the academic guideline for the freshmans 
\end{itemize}

\end{section}

\begin{section}{SYNERGISTIC ACTIVITIES}
    
\begin{itemize}[leftmargin=1.5em]
    \item Talks: 
    \begin{itemize}[leftmargin=1.5em]
        \item AAS 243rd Meeting, Expected Jan 2024, LA, US
        \item NCSA, Jul 2023, IL, US
    \end{itemize}
    \newpage
    \item Conferences: 
    \begin{itemize}[leftmargin=1.5em]
        \item AAS 243rd Meeting, \textbf{Oral presenter}, Expected Jan 2024, LA, US
        \item STEM Career Exploration and Symposium at UIUC, \textbf{Poster Presenter}, Jul 2023, IL, US
        \item The Transient and Variable Universe Conference at NCSA, Jun 2023, IL, US
        \item AAS 241st Meeting, Jan 2023, WA, US
        \item EU Contest for Young Scientists, \textbf{Poster Presenter}, Sep 2019, Sofia, Bulgaria
    \end{itemize}
    \item Was student in: 
    \begin{itemize}[leftmargin=1.5em]
        \item University of California Berkeley (4.000/4.000), Summer 2022
        \item Shanghai Jiao Tong University (4.00/4.00), Summer 2021
    \end{itemize}
    \item Is member of: 
    \begin{itemize}[leftmargin=1.5em]
        \item LSST Dark Energy Science Collaboration
    \end{itemize}
\end{itemize}

\end{section}

\begin{section}{PROFESSIONAL EXPERIENCES}

\textbf{Astrophotographer} at \textit{Personal 25-centimeter Remote Observatory (Hebei, China)} \hfill \textbf{Jan 2018 - Jul 2022} 
\begin{itemize}[leftmargin=1.5em]
    \item Identified a suitable site in Hebei, China and built a 2$\times$2-meter storage facility with internet access and a retractable roof
    \item Selected, assembled, and tested the equipment, and successfully developed a remotely-operated facility
    \item Regularly captured emission nebulae and selected photos are listed in \href{https://cheysen.fit/astrophotography/}{my website}
\end{itemize}

\textbf{Director} of \textit{Physics Society, University of Macau (Macau SAR, China)} \hfill \textbf{Aug 2020 - Feb 2021} 
\begin{itemize}[leftmargin=1.5em]
    \item Founded the University's physics society and promoted its activities on social media platforms
    \item Significantly expanded the Society's membership through effective outreach activities and university club fairs
    \item Organized and led a team of undergraduate students at the 2020 China Undergraduate Physics Tournament
\end{itemize}

\textbf{Student Helper} at \textit{Department of Physics and Chemistry, University of Macau (Macau SAR, China)} \hfill \textbf{Jul 2020 - Oct 2020} 
\begin{itemize}[leftmargin=1.5em]
    \item Requested equipmrnt quotes and negotiated contracts with suppliers for physics-related research 
    \item Tracked the department's procurement budget and developed budgeting recommendations
    \item Purchased supplies for the 2020 China Undergraduate Physics Tournament
\end{itemize}

\textbf{Organizer} of \textit{Beijing Astronomy and Astrophysics Olympiad (Beijing \& Guangdong, China)} \hfill \textbf{Jan 2018 - Apr 2018} 
\begin{itemize}[leftmargin=1.5em]
    \item Coordinated the preparations for the 2018 Olympiad with participating high schools across China
    \item Invited distinguished professors and organized guest lectures/workshops on astronomy and
    astrophysics
    \item Developed the competition paper , prepared and purchased competition materials, and supported panel judges with evaluations
    \item Led the Awards Ceremony at the National Astronomical Observatory
\end{itemize}

\textbf{President} of \textit{Beijing Youth Astronomy Union (Beijing, China)} \hfill \textbf{Aut 2017 - Aug 2018} 
\begin{itemize}[leftmargin=1.5em]
    \item Held a series of lectures on astrophysics and prepared supplementary to train candidates for the National Astronomy Olympiad
    \item Held roadside observation events near the Olympic Park, Beijing
    \item Attended the Ninth National Astronomical Society Development Forum in Weihai, Shandong
    \item Operated the WeChat public account “北京市中学生天文联盟”, and the most popular post obtained more than 100,000 readings
\end{itemize}

\end{section}

\begin{section}{TECHNICAL SKILLS}

\begin{tabular}{rl}
    Skilled in: & Python, \LaTeX, MATLAB, Git, and Shell Bash/Zsh \\
    Basic Knowledge: & C/C++, Mathematica, Julia, docker, SQL, and Java \\
    Softwares: & MaxIm DL, COMSOL, Altium Designer, KiCAD, Solidworks, Cinema 4D, and SPSS \\
    Often-used Packages: & \href{https://www.astropy.org/}{\texttt{AstroPy}}, \href{https://github.com/pmelchior/scarlet}{\texttt{Scarlet}}, \href{https://pytorch.org/}{\texttt{PyTorch}}, \href{https://github.com/facebookresearch/detectron2}{\texttt{Detectron2}}, and \href{https://cigale.lam.fr/}{\texttt{CIGALE}} \\
    Contributions: & \href{https://github.com/burke86/deepdisc}{\texttt{DeepDISC}}: Using deep learning for classification on astornomical survey images \\
    & \href{https://github.com/Chisen-Lupus/metspec}{\texttt{Metspec}}: Auto-detection and photometry of meteor slitless spectrum \\
    & \href{https://github.com/Chisen-Lupus/DES-SED-fitting}{\texttt{DES-SED-Fitting}}: SED fitting and classification of DES sources \\
    & \href{https://github.com/gnarayan/decat_pointings}{\texttt{DECat-pointings}}: working repository of DECam \\
    & \href{https://github.com/burke86/dwarf_agn_cosmos}{\texttt{Dwarf-AGN-COSMOS}}: Spectral analysis for dwarf AGN candidates in COSMOS field
\end{tabular}

\end{section} 

\end{document}


% 加talk conference