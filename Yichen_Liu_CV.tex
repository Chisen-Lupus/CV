\documentclass[11pt]{article} % minimum 10pt
\usepackage[slantfont, boldfont]{xeCJK}
\usepackage{amsmath}
\usepackage{amssymb}
\usepackage{textcomp}
\usepackage{enumitem}
\usepackage[left=0.8in,top=0.8in,right=0.8in,bottom=0.8in, letterpaper]{geometry}
\usepackage{hyperref}
\usepackage[x11names]{xcolor}
\hypersetup{
    colorlinks=true,
    linkcolor=blue,
    filecolor=blue,
    citecolor=black,      
    urlcolor=Blue3, % url color
    }
\linespread{1.1}%行距 1
% \setlength{\parskip}{1ex}%段距 1.5ex
\setlength{\parindent}{0em}%缩进 0em
\usepackage{fancyhdr}
\pagestyle{fancy}
\fancyhf{}
\renewcommand{\footrulewidth}{0.4pt}
\lfoot{\footnotesize Yichen Liu}
\cfoot{\footnotesize \today}
\rfoot{\footnotesize \thepage \ OF \pageref*{LastPage}}
\renewcommand\headrulewidth{0pt}
% \setlength{\headheight}{0in}%固定页眉位置 页眉与正文baseline的高度差 加上之后其余页就同步了
\setlength{\footskip}{4ex}%固定正文结尾 结尾下限和页脚下限的高度差 加上之后其余页就同步了
% \setlength{\headsep}{0in}
\usepackage{titlesec}
% \titleformat{\section}{\vspace{-3ex}\bf}{}{0em}{}[\hrule height 0.8pt\vspace{-1.5ex}]
\titleformat{\section}{\vspace{-1.75ex}\bf}{}{0em}{}[\hrule height 0.8pt\vspace{-1.25ex}]
\usepackage{lastpage}
\usepackage{fontspec}
% \setmainfont{Garamond}
\setmainfont{Cambria}[Ligatures=TeX]
% \setmainfont{Georgia}
% \setmainfont{Book Antiqua}
\usepackage{enumitem}
% \setenumerate[1]{itemsep=0pt,partopsep=0pt,parsep=\parskip,topsep=5pt}
\setenumerate[1]{itemsep=0pt,partopsep=0pt,parsep=\parskip,topsep=0ex}
\setitemize[1]{itemsep=0ex,partopsep=0ex,parsep=0ex,topsep=0ex}
\setitemize[2]{itemsep=0ex,partopsep=0ex,parsep=0ex,topsep=0ex}
% \setdescription{itemsep=0pt,partopsep=0pt,parsep=\parskip,topsep=5pt}
\usepackage{academicons}
\newcommand{\orcid}[1]{\href{https://orcid.org/#1}{\textcolor[HTML]{A6CE39}{\aiOrcid}}}
\newcommand{\googlescholar}[1]{\href{https://scholar.google.com.hk/citations?user=#1}{\textcolor[HTML]{3983FE}{\aiGoogleScholar}}}
\usepackage{fontawesome5}
\newcommand{\github}[1]{\href{https://github.com/#1}{\textcolor[HTML]{000000}{\faGithub}}}
\usepackage{booktabs}% http://ctan.org/pkg/booktabs
\newcommand{\tabitem}{~~\llap{\textbullet}~~}
\usepackage{multicol}
\usepackage{soul}
% % for narrower code font
% \usepackage{lmodern}
% % Set the default font family to typewriter
% \renewcommand*\familydefault{\ttdefault}
% \usepackage[T1]{fontenc}

% for testing purpose
\DeclareRobustCommand{\yichen}[1]{{\sethlcolor{lime}\hl{#1}}}
% \DeclareRobustCommand{\yichen}[1]{#1}


\begin{document}

\begin{center}\textbf{\Large{YICHEN LIU 刘亦辰}}\end{center}

% \vspace{-2ex}

\begin{tabular}{@{}p{0.05\linewidth}@{}p{0.25\linewidth}@{}p{0.05\linewidth}@{}p{0.65\linewidth}}
    \faPhone & +1-(447)-902-2638 &
        \orcid{0000-0003-4247-0169} & \href{https://orcid.org/0000-0003-4247-0169}{0000-0003-4247-0169}\\
    \faEnvelopeSquare & \href{mailto:yl127@illinois.edu}{yl127@illinois.edu} &
        \github{Chisen-Lupus} & \href{https://github.com/Chisen-Lupus}{Chisen-Lupus} \\
    \faGlobe & \href{https://yliu.fit}{https:/\!/yliu.fit} & 
        \googlescholar{GRjhRLUAAAAJ} & \href{https://scholar.google.com.hk/citations?user=GRjhRLUAAAAJ}{Yichen Liu} \\
    \,\faMapPin & Tucson, AZ, 85719 & 
        \href{https://ui.adsabs.harvard.edu/public-libraries/lSSV4SVjSrmt-qgqILgTcA}{\textcolor[HTML]{0e46a1}{\aiADS}} & \href{https://ui.adsabs.harvard.edu/public-libraries/lSSV4SVjSrmt-qgqILgTcA}{\texttt{=author:("liu,yichen") keyword:("astrophysics")}}
\end{tabular}

\begin{section}{EDUCATION}

\textbf{University of Arizona} | Steward Observatory \hfill AZ, US
\begin{itemize}[leftmargin=1.5em]
    \item Ph.D. Student (Advisor: \href{mailto:xfan@arizona.edu}{Professor Xiaohui Fan}) \hfill \textit{Expected 2024/08}
\end{itemize}
\textbf{University of Illinois Urbana-Champaign} | College of Liberal Arts \& Sciences \hfill IL, US
\begin{itemize}[leftmargin=1.5em]
    \item \textit{Bachelor of Science} (3.95/4.00) \hfill \textit{2022/08 - 2024/05}
    \item Thesis: \textit{Host galaxy properties of HSC-SSP variable AGNs in the COSMOS field} (Advisor: \href{mailto:xinliuxl@illinois.edu}{Professor Xin Liu})
    \item Majors: \textbf{Astrophysics} and \textbf{Mathematics}; Minors: \textbf{Physics}, \textbf{Computer Science}, and \textbf{Chemistry}
    \item Graduated with \textbf{Highest Distinction} in both majors, \textbf{Cum Laude}, and \textbf{James Scholar Honor} 
\end{itemize}
\textbf{University of Macau} | Faculty of Science and Technology \hfill Macau SAR, China
\begin{itemize}[leftmargin=1.5em]
    \item Major: \textbf{Applied Physics and Chemistry} (with \textbf{Honour}); Minor: \textbf{Sociology} \hfill \textit{2019/08 - 2022/05}  
\end{itemize}

\end{section}

\begin{section}{RSEARCH ASSISTANTSHIPS}

\textbf{University of Arizona} | Steward Observatory \hfill AZ, US 
\begin{itemize}[leftmargin=1.5em]
\item Advisor: \href{mailto:xfan@arizona.edu}{Professor Xiaohui Fan} and \href{mailto:egami@arizona.edu}{Professor Eiichi Egami}\hfill \textit{Expected 2024/08}
\end{itemize}

\textbf{Peking University} | The Kavli Institute for Astronomy and Astrophysics \hfill Beijing, China 
\begin{itemize}[leftmargin=1.5em]
\item Advisor: \href{mailto:lho.pku@gmail.com}{Professor Luis C. Ho} \hfill \textit{2024/05 - Present}
\end{itemize}

\textbf{University of Illinos Urbana-Champaign} | Department of Astronomy \hfill IL, US 
\begin{itemize}[leftmargin=1.5em]
\item Advisor: \href{mailto:xinliuxl@illinois.edu}{Professor Xin Liu} \hfill \textit{2022/09 - 2024/05}
\end{itemize}

\textbf{Chinese Academy of Sciences} | National Astronomical Observatory of China \hfill Beijing, China 
\begin{itemize}[leftmargin=1.5em]
\item Advisor: \href{mailto:chjwu@bao.ac.cn}{Professor Chaojian Wu} \hfill \textit{2022/06 - 2022/08} 
\end{itemize}

\textbf{University of Macau} | Institute of Applied Physics and Materials Engineering \hfill Macau SAR, China
\begin{itemize}[leftmargin=1.5em]
\item Advisor: \href{mailto:hcliu@um.edu.mo}{Professor Hongchao Liu} \hfill \textit{2019/09 - 2022/05}
\end{itemize}

\end{section}

\begin{section}{RESEARCH AREAS}

\begin{itemize}[leftmargin=1.5em]
    \item Topics: Supermassive Black Holes, Galaxies, Ghost Imaging
    \item Methods: Model Fitting, Observation, Machine Learning 
\end{itemize}

\end{section}

% This section uses Python code to get real-time information from ADS
% --shell-escape is needed when compiling
% `requests` and `urllib` needed in the code
% compilation time will be greatly increased by including these commands

\begin{section}{PUBLICATIONS}

Summary: \input{ |python get_pubstat.py --ads_token Gh2X1K7QxOgpUq0kU3eG3QWNuer5wV6CaIjgkGdp --library_id lSSV4SVjSrmt-qgqILgTcA --name Yichen}
    
\textbf{First-Author Publications}
\begin{enumerate}[leftmargin=1.5em]
    \input{ |python get_publist.py --ads_token Gh2X1K7QxOgpUq0kU3eG3QWNuer5wV6CaIjgkGdp --library_id lSSV4SVjSrmt-qgqILgTcA --name Yichen --first_author True} 
\end{enumerate}

\textbf{Co-Author Publications}
\begin{enumerate}[leftmargin=1.5em]
    \input{ |python get_publist.py --ads_token Gh2X1K7QxOgpUq0kU3eG3QWNuer5wV6CaIjgkGdp --library_id lSSV4SVjSrmt-qgqILgTcA --name Yichen --first_author False} 
\end{enumerate}

\end{section}

\begin{section}{COLLABORATIONS}

\begin{tabular}{@{}p{0.2\linewidth}@{}p{0.8\linewidth}@{}}
    Member &LSST Dark Energy Science Collaboration \\
\end{tabular}

\end{section}

\begin{section}{AWARDS}

\textbf{Honors and Prizes}

\begin{itemize}[leftmargin=1.5em]
    \item University of Illinois Dean's Honor List (2022-2024) \hfill \textit{2024/07}
    \item University of Macau Dean's Honour List (2020 and 2022) \hfill \textit{2022/08}
    \item Third Prize, China Undergraduate Physics Tournament \hfill \textit{2020/10}
    \item National Team Leader at the 2019 European Union Contest for Young Scientists \hfill \textit{2019/09}
    \item Bronze Medal, International Olympiad of Astronomy and Astrophysics \hfill \textit{2018/11}
    \item First Prize, China Adolescents' Science and Technology Innovation Contest \hfill \textit{2018/08}
    \item Second Prize, China National Astronomy Olympiad \hfill \textit{2018/05}
\end{itemize}
        
\textbf{Scholarships and Grants}

\begin{itemize}[leftmargin=1.5em]
    \item AAS 243rd Meeting Travel Grants \hfill \textit{2023/10}
    \item NCSA SPIN Internship (Summer 2023 \& Academic Year 23-24) \hfill \textit{2023/08}
    \item Smart Star Scholarship for the Summer Program \hfill \textit{2022/06}
    \item Residential College Scholarship for the Summer Program \hfill \textit{2021/05}
    \item University of Macau Full Scholarship (2019-2021) \hfill \textit{2019/08}
\end{itemize}
        
\end{section}

\begin{section}{OBSERVATION EXPERIENCES}

\begin{tabular}{@{}p{0.6\linewidth}@{}p{0.4\linewidth}@{}}
    Cerro Tololo Inter-American Observatory, 4m/DECam & 3 nights (remote) \hfill \textit{2023/01 - 2023/04} \\
    Personal Remote Observatory at Hebei, China, 10in/QHY9sm & \href{https://yliu.fit/astrophotography/}{astrophotography} \hfill \textit{2019/08 - 2022/08}
\end{tabular}

\end{section}

\begin{section}{COLLOQUIA AND SEMINARS}

\begin{itemize}[leftmargin=1.5em]
    \item NCSA lighning talk (IL, US) \hfill \textit{2023/07} 
\end{itemize}

\end{section}

\begin{section}{PRESENTATIONS IN MEETINGS AND CONFERENCES}

\begin{itemize}[leftmargin=1.5em]
    \item Poster, \textit{2nd Annual NCSA Student Research Conference} (IL, US): ``Detection, Instance Segmentation, and Classification for Astronomical Surveys with Deep Learning'' \hfill \textit{2024/04} 
    \item Talk, \textit{243th Meeting of the American Astronomical Society} (LA, US): ``Host galaxy properties of variable AGNs in COSMOS and prospects for LSST'' \hfill \textit{2024/01} 
    \item Poster, \textit{STEM Career Exploration and Symposium} (IL, US) \hfill \textit{2023/07} 
    \item Poster, \textit{EU Contest for Young Scientists} (Sofia, Bulgaria): ``Narrow-Band Photometry of Emission Nebulae Using Small-Caliber Telescope'' \hfill \textit{2019/09} 
\end{itemize}

\end{section}

\begin{section}{TEACHING EXPERIENCES}

University of Illinois at Urbana-Champaign | Department of Astronomy \hfill IL, US
\begin{itemize}[leftmargin=1.5em]
    \item Undergraduate Tutor for PHYS 213, 214, and ASTR 210 \hfill \textit{2023/01 - 2023/05}
\end{itemize}

University of Macau | Unofficial Seminar of Physics \hfill Macau SAR, China
\begin{itemize}[leftmargin=1.5em]
    \item Crafted and taught ``\href{https://github.com/Chisen-Lupus/Seminar-of-Physics-UM/blob/main/SPUM%20102%20The%20tools%20of%20physical%20tool.pdf}{SPUM 102 The tools of the physical tools}'' (\href{https://www.youtube.com/watch?v=nQkv03r-XeQ&list=PLV9fHDZW7hHWQ9rrAk7c9kdeV-Lqyt7pV&index=10}{video}), delivering in-depth explorations of mathematical concepts beyond the standard undergraduate curriculum \hfill \textit{2022/02 - 2022/05}
\end{itemize}

\end{section}

\begin{section}{OUTREACH}

Physics and Mathematics Video Creator on Bilibili \hfill \textit{2021/09 - 2022/08}
\begin{itemize}[leftmargin=1.5em]
    \item \href{https://www.bilibili.com/video/BV1th411W7xu/}{Complex Variables Video Series} (in Chinese, currently over 160,000 views) 
\end{itemize}

President of Beijing Youth Astronomy Union \hfill \textit{2017/08 - 2018/08} 
\begin{itemize}[leftmargin=1.5em]
    \item Sidewalk astronomy near Beijing's Olympic Park (organizer)
    \item WeChat Official Account ``北京市中学生天文联盟'' (manager, with a post over 100,000 views)
    \item 2nd Beijing Astronomy and Astrophysics Olympiad (organizer) 
\end{itemize}

\end{section}

\begin{section}{DEPARTMENTAL SERVICES}
    
\begin{itemize}[leftmargin=1.5em]
    \item Director of the Physics Society, University of Macau \hfill \textit{2020/08 - 2021/02}
    \item Student Helper of the Department of Physics and Chemistry, University of Macau \hfill \textit{2020/07 - 2020/10} 
\end{itemize}

\end{section}

% \begin{section}{RESEARCH MENTEES ADVISED/CO-ADVISED}

% \begin{tabular}{@{}p{0.2\linewidth}@{}p{0.8\linewidth}@{}}
%     Diego Miura & Yale University \hfill \textit{2024-Present} \\
% \end{tabular}

% \end{section}

\begin{section}{PROFESSIONAL SKILLS}

\begin{tabular}{@{}p{0.2\linewidth}@{}p{0.8\linewidth}@{}}
    \textbf{Programming}: & Python/Jupyter, \LaTeX, MATLAB, Git, Arduino, Bash/Zsh, C/C++, Mathematica, Julia, Docker, SQL, and Java \\
    \textbf{Softwares}: & COMSOL, Altium Designer, KiCAD, Solidworks, and Cinema 4D \\
\end{tabular}
    
\end{section} 

\end{document}

