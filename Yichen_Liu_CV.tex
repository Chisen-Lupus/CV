\documentclass[10pt]{article} % minimum 10pt
\usepackage[slantfont, boldfont]{xeCJK}
\usepackage{amsmath}
\usepackage{amssymb}
\usepackage{textcomp}
\usepackage{enumitem}
\usepackage[left=0.4in,top=0.4in,right=0.4in,bottom=0.5in, letterpaper]{geometry}
\usepackage{hyperref}
\usepackage[x11names]{xcolor}
\hypersetup{
    colorlinks=true,
    linkcolor=blue,
    filecolor=blue,
    citecolor=black,      
    urlcolor=Blue3, % url color
    }
\linespread{0.97}%行距 1
% \setlength{\parskip}{1.5ex}%段距 1.5ex
\setlength{\parindent}{0em}%缩进 0em
\usepackage{fancyhdr}
\pagestyle{fancy}
\fancyhf{}
\renewcommand{\footrulewidth}{0.4pt}
\lfoot{\small Yichen Liu}
\cfoot{\small \today}
\rfoot{\small \thepage\hspace{1pt} OF \pageref*{LastPage}}
\renewcommand\headrulewidth{0pt}
% \setlength{\headheight}{0in}%固定页眉位置 页眉与正文baseline的高度差 加上之后其余页就同步了
\setlength{\footskip}{4ex}%固定正文结尾 结尾下限和页脚下限的高度差 加上之后其余页就同步了
% \setlength{\headsep}{0in}
\usepackage{titlesec}
\titleformat{\section}{\vspace{-2.75ex}\bf}{}{0em}{}[\hrule height 0.8pt\vspace{-1.5ex}]
\usepackage{lastpage}
\usepackage{fontspec}
% \setmainfont{Garamond}
\setmainfont{Cambria}
% \setmainfont{Georgia}
% \setmainfont{Book Antiqua}
\usepackage{enumitem}
\setenumerate[1]{itemsep=0pt,partopsep=0pt,parsep=\parskip,topsep=5pt}
\setitemize[1]{itemsep=0ex,partopsep=0ex,parsep=0ex,topsep=0ex}
\setitemize[2]{itemsep=0ex,partopsep=0ex,parsep=0ex,topsep=0ex}
% \setdescription{itemsep=0pt,partopsep=0pt,parsep=\parskip,topsep=5pt}
\usepackage{academicons}
\newcommand{\orcid}[1]{\href{https://orcid.org/#1}{\textcolor[HTML]{A6CE39}{\aiOrcid}}}
\newcommand{\googlescholar}[1]{\href{https://scholar.google.com.hk/citations?user=#1}{\textcolor[HTML]{3983FE}{\aiGoogleScholar}}}
\usepackage{fontawesome}
\newcommand{\github}[1]{\href{https://github.com/#1}{\textcolor[HTML]{000000}{\faGithub}}}
\usepackage{booktabs}% http://ctan.org/pkg/booktabs
\newcommand{\tabitem}{~~\llap{\textbullet}~~}
\usepackage{multicol}
\usepackage{soul}

% for testing purpose
\DeclareRobustCommand{\yichen}[1]{{\sethlcolor{lime}\hl{(YL:) #1}}}
% \DeclareRobustCommand{\yichen}[1]{#1}

\begin{document}

\begin{center}\textbf{\large{YICHEN LIU}}\end{center}

\vspace{-2ex}

\begin{multicols}{2}

\begin{tabular}{rl}
    \orcid{0000-0003-4247-0169} & \href{https://orcid.org/0000-0003-4247-0169}{0000-0003-4247-0169} \\
    \github{Chisen-Lupus} & \href{https://github.com/Chisen-Lupus}{Chisen-Lupus} \\
    \googlescholar{GRjhRLUAAAAJ} & \href{https://scholar.google.com.hk/citations?user=GRjhRLUAAAAJ}{Yichen Liu} \\
\end{tabular}

\begin{tabular}{rl}
    tel & +1-(447) 902-2638 \\
    web & \href{https://yliu.fit}{https:/\!/yliu.fit} \\
    e-mail & \href{mailto:yl127@illinois.edu}{yl127@illinois.edu} \\
\end{tabular}

\end{multicols}

\begin{section}{EDUCATION}

\textbf{University of Illinois at Urbana-Champaign} | College of Liberal Arts \& Sciences \hfill \textbf{2022/08 - Present}
\begin{itemize}[leftmargin=1.5em]
    \item Bachelor of Science (\textbf{Honor}) in \textbf{Astrophysics} and \textbf{Mathematics}, expected 2024/05 \hfill \textbf{4.00/4.00} 
    \item Minor in \textbf{Physics}, \textbf{Computer Science}, and \textbf{Chemistry}
    % \item Enrolling in the \textbf{LAS Honors program} 
\end{itemize}
\textbf{University of Macau} | Faculty of Science and Technology \hfill \textbf{2019/08 - 2022/05}
\begin{itemize}[leftmargin=1.5em]
    \item Completed Junior Year of \textbf{Applied Physics and Chemistry (Honour)}  %\hfill \textbf{3.40/4.00} (last year: 3.81)
    % \item Enrolled in the \textbf{Honours College}
    % \item Director of \textbf{University of Macau Physics Society} \hfill Aug 2020 - Feb 2021
\end{itemize}
    
\end{section}

\begin{section}{PUBLICATIONS AND ABSTRACTS}
    
\begin{enumerate}[leftmargin=1.5em]
    \item \yichen{\textbf{Yichen Liu}, et al., ``Black hole - host galaxy relations of dwarf AGNs in DES C3, X3, and E2 fields up to $z\sim???$'', In prep. \\ (It might takes 2 lines)}
    \item \textbf{Yichen Liu}, Colin J. Burke, Charlotte A. Ward, Xin Liu, Priya Natarajan, ``Host galaxy properties of HSC-SSP variable AGNs in the COSMOS field and expectations for Rubin Observatory'', American Astronomical Society Meeting \#243, id. 3936
    \item Grant Merz, \textbf{Yichen Liu}, Colin J. Burke, Patrick D. Aleo, Xin Liu, Matias Carrasco Kind, Volodymyr Kindratenko, Yufeng Liu, \href{https://academic.oup.com/mnras/advance-article-abstract/doi/10.1093/mnras/stad2785/7273850?redirectedFrom=fulltext}{``Detection, Instance Segmentation, and Classification for Astronomical Surveys with Deep Learning (DeepDISC): Detectron2 Implementation and Demonstration with Hyper Suprime-Cam Data,"} MNRAS 526, 1122 (2023)
    \item \textbf{Yichen Liu}, Peixia Zheng, and Hong-Chao Liu, \href{https://opg.optica.org/oe/fulltext.cfm?uri=oe-30-9-14073&id=471300}{``Anti-loss-compression image encryption based on computational ghost imaging using discrete cosine transform and orthogonal patterns,"} Optics Express 30, 14073 (2022)
    \item Peixia Zheng, \textbf{Yichen Liu}, and Hong-Chao Liu, \href{http://www.irla.cn/cn/article/doi/10.3788/IRLA20211058}{``Single-pixel imaging and metasurface imaging,"} Infrared and Laser Engineering 50, 20211058-1 (2022)
\end{enumerate}

\end{section}

\begin{section}{RSEARCH EXPERIENCES}

\textbf{Research Assistant} (advisor: \href{mailto:xinliuxl@illinois.edu}{Professor Xin Liu}), \textit{Department of Astronomy} \hfill \textbf{2022/09 - Present} 
\begin{itemize}[leftmargin=1.5em]
    \item \textbf{Project 1: instance segmentation in astronomical surveys using machine learning} (NCSA SPIN internship) \hfill $\rceil$ %2022/09 - Present 
    \begin{itemize}[leftmargin=1.5em]
        \item Evaluated and optimized source extraction pipelines for \href{https://github.com/burke86/deepdisc}{\texttt{DeepDISC}} and \href{https://github.com/burke86/astro_rcnn}{\texttt{Astro R-CNN}} using \href{https://github.com/kbarbary/sep/tree/v1.1.x}{\texttt{Sep}} and \href{https://github.com/pmelchior/scarlet}{\texttt{Scarlet}} frameworks
        \item Orchestrated simulations employing diverse models and configurations on PhoSim data via \href{https://www.ncsa.illinois.edu/research/project-highlights/hal-cluster/}{HAL cluster} with \href{https://github.com/facebookresearch/detectron2}{\texttt{Detectron2}}
        \item Enhanced the pipeline by incorporating Transformer models, MViT and VitDet, to improve instance segmentation
    \end{itemize}
    \item \textbf{Project 2: DES SED fitting} \hfill $\rceil$ %2023/02 - 2023/05
    \begin{itemize}[leftmargin=1.5em]
        \item Performed SED fitting on sources cataloged in DES and WISE using the \href{https://cigale.lam.fr/}{\texttt{CIGALE}} toolkit
        \item Developed selection criteria and identified AGN candidates from extensive source catalogs
    \end{itemize}
    \item \textbf{Project 3: host galaxy properties of variable AGNs in the COSMOS field} \hfill $\rceil$ %2023/06 - Present
    \begin{itemize}[leftmargin=1.5em]
        \item Cross-referenced variability-selected AGNs from HSC DR2 to subsequent DR3, SIMBAD, DESI, and COSMOS2020
        \item Created scripts for batch downloading of optical spectra from various sources such as SDSS, zCOSMOS, Magellan, DEIMOS, among others, and econciled discrepancies in spectral data across different databases
        \item Determined black hole mass - host galaxy mass relation through SED fitting
        \item Investigated star formation main sequence using \href{https://github.com/legolason/PyQSOFit}{\texttt{PyQSOFit}} toolkit
    \end{itemize}
    \item \yichen{\textbf{Project 3.1: Black hole - host galaxy relations of dwarf AGNs in DES field} \hfill $\rceil$} % 2023/06 - Present
    \begin{itemize}[leftmargin=1.5em]
        \item \yichen{Matched dwarf AGN candidates in the DES C3, X3, and E2 fields to ??? catalogs}
        \item \yichen{Will perform a simple PyQSOFit pipeline on this dataset}
    \end{itemize}
    \item \yichen{\textbf{Project 4: redshift estimation in astronomical surveys using machine learning} (NCSA SPIN internship) \hfill $\rceil$} %2023/10 - Present
    \begin{itemize}[leftmargin=1.5em]
        \item \yichen{Will resolve issues in DeepDisc in LINCC branch}
    \end{itemize}
\end{itemize}

\textbf{Summer Research Internship} (advisor: \href{mailto:chjwu@bao.ac.cn}{Professor Chaojian Wu}), \textit{National Astronomical Observatory of China} \hfill \textbf{2022/06 - 2022/08} 
\begin{itemize}[leftmargin=1.5em]
    \item \textbf{Project: meteor slitless spectrum} \hfill $\rceil$
    \begin{itemize}[leftmargin=1.5em]
        \item Analyzed the spectral data of 2021 Gemini meteors captured with DSLR cameras equipped with diffraction gratings
        \item Employed Python to dissect the intensities of Sodium and Magnesium lines 
        \item Developed machine learning algorithms for the automated detection and photometric assessment of meteor recordings
    \end{itemize}
\end{itemize}

\textbf{Research Assistant} (advisor: \href{mailto:hcliu@um.edu.mo}{Professor Hongchao Liu}), \textit{Institute of Applied Physics \& Materials Engineering}\hfill \textbf{2019/09 - 2022/05} 
\begin{itemize}[leftmargin=1.5em]
    \item \textbf{Project 1: ghost imaging in complex environment} \hfill $\rceil$ %Aug 2019 - Aug 2021
    \begin{itemize}[leftmargin=1.5em]
        \item Conducted a comprehensive review of contemporary ghost imaging and single-pixel imaging research
        \item Assessed the quality of ghost imaging across various setups using MATLAB, producing reports and comparative analyses
        \item Compared the reflection patterns between distorting and standard mirrors to inform imaging technique improvements
    \end{itemize}
    \item \textbf{Project 2: anti-loss image encryption based on ghost imaging} \hfill $\rceil$ %Sep 2021 - Feb 2022
    \begin{itemize}[leftmargin=1.5em]
        \item Executed experimental research into ghost imaging, investigating the potential of metamaterials and metasurfaces
        \item Created Python algorithms leveraging compressive sensing and gradient descent methodologies.
        \item Managed and fine-tuned computational imaging simulations on \href{https://pytorch.org/}{\texttt{PyTorch}} with high-performance GPUs
        \item Published a first-authored research paper as the first undergraduate student in the department, and \href{https://www.tdm.com.mo/en/news-detail/683438?isvideo=false&lang=en&category=all}{exposed by local media} 
    \end{itemize}
    \item \textbf{Project 3: ghost imaging using recurrent neural network} \hfill $\rceil$ %Mar 2022 - May 2022
    \begin{itemize}[leftmargin=1.5em]
        \item Collaborated with a team of postgraduate students in managing laser equipment for experimental setups
        \item Validated existing ghost imaging techniques incorporating neural networks
        \item Developed Python pipelines integrating recurrent and convolutional neural network architectures for ghost imaging
    \end{itemize}
\end{itemize}

\end{section}

\begin{section}{SYNERGISTIC ACTIVITIES}

\begin{tabular}{@{}p{0.15\linewidth}@{}p{0.85\linewidth}@{}}
    \textbf{Summer schools}: & University of California Berkeley (4.000/4.000), 2022 \hfill $\rceil$\\
    & Shanghai Jiao Tong University (4.00/4.00), 2021 \\
    \textbf{Presentations}: & AAS 243rd Meeting, \textbf{Oral presenter}, Scheduled Jan 2024, LA, US \hfill $\rceil$\\
    & STEM Career Exploration and Symposium, \textbf{Poster Presenter}, Jul 2023, IL, US \\
    & NCSA lighning talk, \textbf{Oral presenter}, Jul 2023, IL, US \\
    & EU Contest for Young Scientists, \textbf{Poster Presenter}, Sep 2019, Sofia, Bulgaria \\
    \textbf{Membership}: &LSST Dark Energy Science Collaboration \hfill $\rceil$\\
\end{tabular}

\end{section}

\begin{section}{OBSERVATION EXPERIENCES}
    
\begin{itemize}[leftmargin=1.5em]
    \item Cerro Tololo Inter-American Observatory, Blanco 4m / DECam: 3 nights observation \hfill 2023/01 - 2023/04
    \item Personal Remote Observatory, BKP250 / QHY9sm: astrophotography and photometry \hfill 2019/08 - 2022/08
\end{itemize}

\end{section}

\begin{section}{AWARDS AND GRANTS}

    \begin{itemize}[leftmargin=1.5em]
        \item AAS 243rd Meeting Travel Grants from Department of Astronomy \hfill 2023/10
        % \item NCSA SPIN Internship (Summer 2023 \& Academic Year 23-24) \hfill 2023/08
        \item University of Illinois Dean's Honor List (2022-2023) \hfill 2023/07
        \item Smart Star Sponsorship for studies at University of California, Berkeley \hfill 2022/06
        \item University of Macau Dean's Honour List (2020 and 2022) \hfill 2022/08
        \item Residential College Summer Programme Sponsorship for studies at Shanghai Jiao Tong university \hfill 2021/05
        % \item Third Prize, China Undergraduate Physics Tournament \hfill 2020/10
        \item National Team Leader at the 2019 European Union Contest for Young Scientists \hfill 2019/09
        \item University of Macau Full Scholarship (2019-2021) \hfill 2019/08
        \item Bronze Medal, International Olympiad of Astronomy and Astrophysics \hfill 2018/11
        % \item First Prize, China Adolescents' Science and Technology Innovation Contest \hfill 2018/08
        % \item Second Prize, Deng Feng National Contest on Science and Innovation \hfill 2018/08
        % \item Second Prize, China National Astronomy Olympiad \hfill 2018/05
    \end{itemize}
        
\end{section}

\begin{section}{TEACHING EXPERIENCES}

\textbf{Physics and Mathematics Video Creator on Bilibili} \hfill \textbf{2021/09 - Present}
\begin{itemize}[leftmargin=1.5em]
    \item Produced and broadcasted  educational content on physics and mathematics to a wide audience on the Bilibili platform, with a focus on undergraduate topics and self-study materials, such as the zeta function
    \item Achieved widespread outreach with \href{https://www.bilibili.com/video/BV1th411W7xu/}{the most popular video} surpassing 160,000 views, contributing to the public education
\end{itemize}

\textbf{Undergraduate Tutor of Department of Astronomy} \hfill \textbf{2023/01 - 2023/05}
\begin{itemize}[leftmargin=1.5em]
    \item Designed and led interactive tutoring sessions for undergraduates majoring in Astronomy, covering foundational concepts in thermal physics, quantum physics, and astrophysics to supplement their introductory courses
\end{itemize}

\textbf{Organizer and Lecturer of Seminar of Physics at the University of Macau} \hfill \textbf{2022/02 - 2022/05}
\begin{itemize}[leftmargin=1.5em]
    \item Founded and co-organized a series of informal but comprehensive lectures with my peer, \href{http://runawayfancy.me/}{Jiheng Duan}, to provide advanced mathematical and physical concepts beyond the University of Macau's curriculum
    \item Addressed gaps in the theoretical understanding necessary for future research in physics, covering topics such as classical mechanics and partial differential equations for Department of Physics and Chemistry students
    \item Authored a comprehensive guide for freshmen, providing a roadmap for academic development and graduate study preparation
    \item Regularly conducted sessions bi-weekly throughout the semester, maintaining a consistent and rigorous teaching schedule
    \item Developed and delivered \textit{SPUM 102 The tools of physical tools}, a lecture series encompassing topics such as complex variables, gamma functions, integral transforms, delta functions, and Green's functions, with a detailed \href{https://github.com/Chisen-Lupus/Seminar-of-Physics-UM/blob/main/SPUM%20102%20The%20tools%20of%20physical%20tool.pdf}{syllabus} provided
    \item Made lecture recordings accessible to the public on \href{https://www.youtube.com/watch?v=nQkv03r-XeQ&list=PLV9fHDZW7hHWQ9rrAk7c9kdeV-Lqyt7pV&index=10}{Youtube}, extending the reach of these resources beyond the classroom
\end{itemize}

\end{section}

\begin{section}{EXTRACURRICULAR EXPERIENCES}
    
\textbf{Astrophotographer}, \textit{Personal 25‑centimeter Remote Observatory (Hebei, China)} \hfill \textbf{2018/01 ‑ 2022/07}
\begin{itemize}[leftmargin=1.5em]
    \item Sourced and developed a 2×2‑meter remote observatory with full internet connectivity and a retractable roof
    \item Curated and calibrated a suite of astronomical equipment and 3D‑printed accessories, which can be fully controlled remotely
    \item Conducted regular astrophotography sessions, capturing images of emission nebulae, with a selection showcased on \href{https://yliu.fit/astrophotography/}{my webpage}.
\end{itemize}

\textbf{Director}, \textit{Physics Society, University of Macau (Macau SAR, China)} \hfill \textbf{2020/08 - 2021/02} 
\begin{itemize}[leftmargin=1.5em]
    \item Established and expanded the Physics Society, leading promotional efforts and significantly growing its membership
    \item Guided undergraduates through the China Undergraduate Physics Tournament, enhancing the society's academic community
\end{itemize}

\textbf{Student Helper}, \textit{Department of Physics and Chemistry, University of Macau (Macau SAR, China)} \hfill \textbf{2020/07 - 2020/10} 
\begin{itemize}[leftmargin=1.5em]
    \item Handled equipment procurement processes and managed budget recommendations, streamlining the department's operations
\end{itemize}

\textbf{President}, \textit{Beijing Youth Astronomy Union (Beijing, China)} \hfill \textbf{2017/08 - 2018/08} 
\begin{itemize}[leftmargin=1.5em]
    \item Conducted educational series on astrophysics, providing Olympiad candidates with additional training resources
    \item Organized public stargazing events adjacent to Beijing’s Olympic Park to foster community engagement in astronomy
    \item Managed the WeChat account ``北京市中学生天文联盟'', achieving widespread readership with posts exceeding 100,000 views
\end{itemize}

\textbf{Organizer}, \textit{Beijing Astronomy and Astrophysics Olympiad (Beijing \& Guangdong, China)} \hfill \textbf{2018/01 - 2018/04} 
\begin{itemize}[leftmargin=1.5em]
    \item Orchestrated the logistical planning of the 2018 Olympiad, liaising with high schools nationwide for participation
    \item Composed the Olympiad's examination materials, orchestrated material procurement, and supported the judging panels
\end{itemize}

\end{section}

\begin{section}{SUMMARY OF TECHNICAL SKILLS}

    \begin{tabular}{rl}
        Programming: & Python, \LaTeX, MATLAB, Git, Arduino, Shell Bash/Zsh, C/C++, Mathematica, Julia, docker, SQL, and Java \\
        Softwares: & MaxIm DL, COMSOL, Altium Designer, KiCAD, Solidworks, Cinema 4D, and SPSS \\
        Python Packages: & \href{https://www.astropy.org/}{\texttt{AstroPy}}, \href{https://github.com/pmelchior/scarlet}{\texttt{Scarlet}}, \href{https://pytorch.org/}{\texttt{PyTorch}}, \href{https://github.com/facebookresearch/detectron2}{\texttt{Detectron2}}, \href{https://cigale.lam.fr/}{\texttt{CIGALE}}, and \href{https://github.com/legolason/PyQSOFit}{\texttt{PyQSOFit}} \\
        Machine Learning: & Neural (RNN, Mask R-CNN, ResNet, and Transformer) \\
        % Contributions: & \href{https://github.com/burke86/deepdisc}{\texttt{DeepDISC}}: Using deep learning for classification on astornomical survey images \\
        % & \href{https://github.com/Chisen-Lupus/metspec}{\texttt{Metspec}}: Auto-detection and photometry of meteor slitless spectrum \\
        % & \href{https://github.com/Chisen-Lupus/DES-SED-fitting}{\texttt{DES-SED-Fitting}}: SED fitting and classification of DES sources \\
        % & \href{https://github.com/gnarayan/decat_pointings}{\texttt{DECat-pointings}}: working repository of DECam \\
        % & \href{https://github.com/burke86/dwarf_agn_cosmos}{\texttt{Dwarf-AGN-COSMOS}}: Spectral analysis for dwarf AGN candidates in COSMOS field
    \end{tabular}
    
    \end{section} 

\end{document}

