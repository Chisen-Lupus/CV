\documentclass[10pt]{article} % minimum 10pt
\usepackage[slantfont, boldfont]{xeCJK}
\usepackage{amsmath}
\usepackage{enumitem}
\usepackage[left=0.4in,top=0.4in,right=0.4in,bottom=0.5in, letterpaper]{geometry}
\usepackage{hyperref}
\usepackage[x11names]{xcolor}
\hypersetup{
    colorlinks=true,
    linkcolor=blue,
    filecolor=blue,
    citecolor=black,      
    urlcolor=Blue3, % url color
    }
\linespread{0.9}%行距 1
% \setlength{\parskip}{1.5ex}%段距 1.5ex
\setlength{\parindent}{0em}%缩进 0em
\usepackage{fancyhdr}
\pagestyle{fancy}
\fancyhf{}
\renewcommand{\footrulewidth}{0.4pt}
\lfoot{Yichen Liu}
\cfoot{\today}
\rfoot{\thepage\hspace{1pt} OF \pageref*{LastPage}}
\renewcommand\headrulewidth{0pt}
% \setlength{\headheight}{0in}%固定页眉位置 页眉与正文baseline的高度差 加上之后其余页就同步了
\setlength{\footskip}{4ex}%固定正文结尾 结尾下限和页脚下限的高度差 加上之后其余页就同步了
% \setlength{\headsep}{0in}
\usepackage{titlesec}
\titleformat{\section}{\vspace{-1.5ex}\bf}{}{0em}{}[\hrule height 0.8pt\vspace{-1.5ex}]
\usepackage{lastpage}
\usepackage{fontspec}
% \setmainfont{Garamond}
\setmainfont{Cambria}
% \setmainfont{Georgia}
% \setmainfont{Book Antiqua}
\usepackage{enumitem}
\setenumerate[1]{itemsep=0pt,partopsep=0pt,parsep=\parskip,topsep=5pt}
\setitemize[1]{itemsep=0ex,partopsep=0ex,parsep=0ex,topsep=0ex}
\setitemize[2]{itemsep=0ex,partopsep=0ex,parsep=0ex,topsep=0ex}
% \setdescription{itemsep=0pt,partopsep=0pt,parsep=\parskip,topsep=5pt}
\usepackage{academicons}
\newcommand{\orcid}[1]{\href{https://orcid.org/#1}{\textcolor[HTML]{A6CE39}{\aiOrcid}}}
\newcommand{\googlescholar}[1]{\href{https://scholar.google.com.hk/citations?user=#1}{\textcolor[HTML]{3983FE}{\aiGoogleScholar}}}
\usepackage{fontawesome}
\newcommand{\github}[1]{\href{https://github.com/#1}{\textcolor[HTML]{000000}{\faGithub}}}
\usepackage{booktabs}% http://ctan.org/pkg/booktabs
\newcommand{\tabitem}{~~\llap{\textbullet}~~}
\usepackage{multicol}
\usepackage{soul}
\DeclareRobustCommand{\yichen}[1]{{\sethlcolor{lime}\hl{(YL:) #1}}}

\begin{document}

\begin{center}\bf{\large{YICHEN LIU}}\end{center}

\begin{multicols}{2}

\begin{tabular}{rl}
    tel & +1-(447) 902-2638 \\
    e-mail & \href{mailto:yl127@illinois.edu}{yl127@illinois.edu} \\
    web & \href{https://yliu.fit}{https:/\!/yliu.fit}
\end{tabular}

\begin{tabular}{rl}
    \orcid{0000-0003-4247-0169} & \href{https://orcid.org/0000-0003-4247-0169}{0000-0003-4247-0169} \\
    \github{Chisen-Lupus} & \href{https://github.com/Chisen-Lupus}{Chisen-Lupus} \\
    \googlescholar{GRjhRLUAAAAJ} & \href{https://scholar.google.com.hk/citations?user=GRjhRLUAAAAJ}{Yichen Liu} \\
\end{tabular}

\end{multicols}

\begin{section}{EDUCATION}

\textbf{University of Illinois at Urbana-Champaign} | College of Liberal Arts \& Sciences \hfill \textbf{Aug 2022 - Present}
\begin{itemize}[leftmargin=1.5em]
    \item Bachelor of Science (\textbf{Honor}) in \textbf{Astrophysics} and \textbf{Mathematics}, expected May 2024 \hfill \textbf{4.00/4.00} 
    \item Minor in \textbf{Physics}, \textbf{Computer Science}, and \textbf{Chemistry}
    % \item Enrolling in the \textbf{LAS Honors program} 
\end{itemize}
\textbf{University of Macau} | Faculty of Science and Technology \hfill \textbf{Aug 2019 - May 2022}
\begin{itemize}[leftmargin=1.5em]
    \item Completed Junior Year of \textbf{Applied Physics and Chemistry (Honour)}  %\hfill \textbf{3.40/4.00} (last year: 3.81)
    % \item Enrolled in the \textbf{Honours College}
    % \item Director of \textbf{University of Macau Physics Society} \hfill Aug 2020 - Feb 2021
\end{itemize}
    
\end{section}

\begin{section}{TECHNICAL SKILLS}

\begin{tabular}{rl}
    Programming: & Python, \LaTeX, MATLAB, Git, Arduino, Shell Bash/Zsh, C/C++, Mathematica, Julia, docker, SQL, and Java \\
    Softwares: & MaxIm DL, COMSOL, Altium Designer, KiCAD, Solidworks, Cinema 4D, and SPSS \\
    Python Packages: & \href{https://www.astropy.org/}{\texttt{AstroPy}}, \href{https://github.com/pmelchior/scarlet}{\texttt{Scarlet}}, \href{https://pytorch.org/}{\texttt{PyTorch}}, \href{https://github.com/facebookresearch/detectron2}{\texttt{Detectron2}}, \href{https://cigale.lam.fr/}{\texttt{CIGALE}}, and \href{https://github.com/legolason/PyQSOFit}{\texttt{PyQSOFit}} \\
    Machine Learning: & Neural (RNN, Mask R-CNN, ResNet, and Transformer) \\
    % Contributions: & \href{https://github.com/burke86/deepdisc}{\texttt{DeepDISC}}: Using deep learning for classification on astornomical survey images \\
    % & \href{https://github.com/Chisen-Lupus/metspec}{\texttt{Metspec}}: Auto-detection and photometry of meteor slitless spectrum \\
    % & \href{https://github.com/Chisen-Lupus/DES-SED-fitting}{\texttt{DES-SED-Fitting}}: SED fitting and classification of DES sources \\
    % & \href{https://github.com/gnarayan/decat_pointings}{\texttt{DECat-pointings}}: working repository of DECam \\
    % & \href{https://github.com/burke86/dwarf_agn_cosmos}{\texttt{Dwarf-AGN-COSMOS}}: Spectral analysis for dwarf AGN candidates in COSMOS field
\end{tabular}

\end{section} 

\begin{section}{PUBLICATIONS AND ABSTRACTS}
    
\begin{enumerate}[leftmargin=1.5em]
    % \item Yichen Liu, et al., DES dwarf AGNs, In prep.
    \item \textbf{Yichen Liu}, Colin J. Burke, Charlotte A. Ward, Xin Liu, Priya Natarajan, ``Host galaxy properties of HSC-SSP variable AGNs in the COSMOS field and expectations for Rubin Observatory'', American Astronomical Society Meeting \#243, id. 3936
    \item Grant Merz, \textbf{Yichen Liu}, Colin J. Burke, Patrick D. Aleo, Xin Liu, Matias Carrasco Kind, Volodymyr Kindratenko, Yufeng Liu, \href{https://academic.oup.com/mnras/advance-article-abstract/doi/10.1093/mnras/stad2785/7273850?redirectedFrom=fulltext}{``Detection, Instance Segmentation, and Classification for Astronomical Surveys with Deep Learning (DeepDISC): Detectron2 Implementation and Demonstration with Hyper Suprime-Cam Data,"} MNRAS 526, 1122 (2023)
    \item \textbf{Yichen Liu}, Peixia Zheng, and Hong-Chao Liu, \href{https://opg.optica.org/oe/fulltext.cfm?uri=oe-30-9-14073&id=471300}{``Anti-loss-compression image encryption based on computational ghost imaging using discrete cosine transform and orthogonal patterns,"} Optics Express 30, 14073 (2022)
    \item Peixia Zheng, \textbf{Yichen Liu}, and Hong-Chao Liu, \href{http://www.irla.cn/cn/article/doi/10.3788/IRLA20211058}{``Single-pixel imaging and metasurface imaging,"} Infrared and Laser Engineering 50, 20211058-1 (2022)
\end{enumerate}

\end{section}

\begin{section}{RSEARCH EXPERIENCES}

\textbf{Research Assistant} at \textit{Department of Astronomy, University of Illinois} \hfill \textbf{Sep 2022 - Present} 
\begin{itemize}[leftmargin=1.5em]
    \item Advisor - \href{mailto:xinliuxl@illinois.edu}{Professor Xin Liu}
    \item Project 1 - instance segmentation in astronomical surveys using machine learning (NCSA SPIN internship) %\hfill Sep 2022 - Present
    \begin{itemize}[leftmargin=1.5em]
        \item Evaluated and optimized source extraction pipelines for \href{https://github.com/burke86/deepdisc}{\texttt{DeepDISC}} and \href{https://github.com/burke86/astro_rcnn}{\texttt{Astro R-CNN}} using \href{https://github.com/kbarbary/sep/tree/v1.1.x}{\texttt{Sep}} and \href{https://github.com/pmelchior/scarlet}{\texttt{Scarlet}} frameworks
        \item Orchestrated simulations employing diverse models and configurations on PhoSim data via the Hardware-Accelerated Learning (HAL) cluster with \href{https://github.com/facebookresearch/detectron2}{\texttt{Detectron2}} integration.
        \item Enhanced the pipeline by incorporating Transformer models, specifically MViT and VitDet, to improve instance segmentation
    \end{itemize}
    \item Project 2 - DES SED fitting %\hfill Feb 2023 - May 2023
    \begin{itemize}[leftmargin=1.5em]
        \item Performed SED fitting on sources cataloged in DES and WISE using the \href{https://cigale.lam.fr/}{\texttt{CIGALE}} toolkit
        \item Developed selection criteria and identified AGN candidates from extensive source catalogs
    \end{itemize}
    \item Project 3 - host galaxy properties of variable AGNs %\hfill Jun 2023 - Present
    \begin{itemize}[leftmargin=1.5em]
        \item Cross-referenced dwarf AGN candidates between HSC DR2 and subsequent catalogs (DR3, SIMBAD, DESI, and COSMOS2020)
        \item Created scripts for batch downloading of optical spectra from various sources such as SDSS, zCOSMOS, Magellan, DEIMOS, among others, and econciled discrepancies in redshift data across different databases
        \item Determined black hole mass - host galaxy mass relation through SED fitting
        \item Investivated star formation main sequence using \href{https://github.com/legolason/PyQSOFit}{\texttt{PyQSOFit}} toolkit
    \end{itemize}
    % \item Project 4 - redshift estimation in astronomical surveys using machine learning (NCSA SPIN internship)
\end{itemize}

\textbf{Summer Research Internship} at \textit{National Observatory of China} \hfill \textbf{Jun 2022 - Aug 2022} 
\begin{itemize}[leftmargin=1.5em]
    \item Advisor - \href{mailto:chjwu@bao.ac.cn}{Professor Chaojian Wu}
    \item Project - meteor slitless spectrum
    \begin{itemize}[leftmargin=1.5em]
        \item Analyzed the spectral data of 2021 Gemini meteors captured with DSLR cameras equipped with diffraction gratings
        \item Employed Python to dissect the intensities of Sodium and Magnesium lines 
        \item Developed machine learning algorithms for the automated detection and photometric assessment of meteor recordings
    \end{itemize}
\end{itemize}

\textbf{Research Assistant} at \textit{Institute of Applied Physics \& Materials Engineering, University of Macau}\hfill \textbf{Aug 2019 - May 2022} 
\begin{itemize}[leftmargin=1.5em]
    \item Advisor - \href{mailto:hcliu@um.edu.mo}{Professor Hongchao Liu}
    \item Project 1 - ghost imaging in complex environment %\hfill Aug 2019 - Aug 2021
    \begin{itemize}[leftmargin=1.5em]
        \item Conducted a comprehensive review of contemporary ghost imaging and single-pixel imaging research
        \item Assessed the quality of ghost imaging across various setups using MATLAB, producing reports and comparative analyses
        \item Investigated the reflection patterns from distorting mirrors, contrasting them with standard mirror reflections to inform imaging technique improvements
    \end{itemize}
    \item Project 2 - anti-loss image encryption based on ghost imaging %\hfill Sep 2021 - Feb 2022
    \begin{itemize}[leftmargin=1.5em]
        \item Executed experimental research into ghost imaging, investigating the potential of metamaterials and metasurfaces
        \item Created Python algorithms leveraging compressive sensing and gradient descent methodologies.
        \item Managed and fine-tuned computational imaging simulations on \href{https://pytorch.org/}{\texttt{PyTorch}} with high-performance GPUs
        \item Published a first-authored research paper as the first undergraduate student in the department, and \href{https://www.tdm.com.mo/en/news-detail/683438?isvideo=false&lang=en&category=all}{exposed by local media} 
    \end{itemize}
    \item Project 3 - ghost imaging using recurrent neural network %\hfill Mar 2022 - May 2022
    \begin{itemize}[leftmargin=1.5em]
        \item Collaborated with a team of postgraduate students in managing laser equipment for experimental setups
        \item Validated existing ghost imaging techniques incorporating neural networks
        \item Developed Python pipelines integrating recurrent and convolutional neural network architectures for ghost imaging
    \end{itemize}
\end{itemize}

\end{section}

\begin{section}{OBSERVATION EXPERIENCE}
    
\begin{itemize}[leftmargin=1.5em]
    \item Cerro Tololo Inter-American Observatory, Blanco 4m / DECam: 3 nights observation \hfill Jan 2023 - Apr 2023
    \item Personal remote observatory, BKP250 / QHY9sm: \href{https://yliu.fit/astrophotography/}{astrophotography} and photometry \hfill Jul 2019 - Aug 2022
\end{itemize}

\end{section}

\begin{section}{SYNERGISTIC ACTIVITIES}

\begin{tabular}{rl}
    Presentations: & AAS 243rd Meeting, \textbf{Oral presenter}, Scheduled Jan 2024, LA, US \\
    & STEM Career Exploration and Symposium, \textbf{Poster Presenter}, Jul 2023, IL, US \\
    & NCSA lighning talk, \textbf{Oral presenter}, Jul 2023, IL, US \\
    & EU Contest for Young Scientists, \textbf{Poster Presenter}, Sep 2019, Sofia, Bulgaria \\
    Summer schools: & University of California Berkeley (4.000/4.000), 2022 \\
    & Shanghai Jiao Tong University (4.00/4.00), 2021 \\
    Membership: &LSST Dark Energy Science Collaboration
\end{tabular}

\end{section}

\begin{section}{AWARDS AND GRANTS}

    \begin{itemize}[leftmargin=1.5em]
        % \item AAS 243rd Meeting Travel Grants, Department of Astronomy \hfill Oct 2023
        % \item NCSA SPIN Internship (Summer 2023 \& Academic Year 23-24) \hfill Aug 2023 
        \item University of Illinois Dean's Honor List (2022-2023) \hfill Jul 2023
        \item Smart Star Sponsorship for studies at University of California, Berkeley \hfill Jun 2022
        \item University of Macau Dean's Honour List (2020 and 2022) \hfill Aug 2022
        \item Residential College Summer Programme Sponsorship for studies at Shanghai Jiao Tong university \hfill May 2021
        \item Third Prize, China Undergraduate Physics Tournament \hfill Oct 2020
        \item National Team Leader at the 2019 European Union Contest for Young Scientists \hfill Sep 2019
        \item University of Macau Full Scholarship (2019-2021) \hfill Aug 2019
        \item Bronze Medal, International Olympiad of Astronomy and Astrophysics \hfill Nov 2018
        \item First Prize, China Adolescents' Science and Technology Innovation Contest \hfill Aug 2018
        % \item Second Prize, Deng Feng National Contest on Science and Innovation \hfill Aug 2018
        % \item Second Prize, China National Astronomy Olympiad \hfill May 2018
    \end{itemize}
        
\end{section}

\begin{section}{TEACHING EXPERIENCE}

\textbf{Undergraduate Tutor of Department of Astronomy} \hfill \textbf{Jan 2023 - May 2023}
\begin{itemize}[leftmargin=1.5em]
    \item Designed and led interactive tutoring sessions for undergraduates majoring in Astronomy, covering foundational concepts in thermal physics, quantum physics, and astrophysics to supplement their introductory courses
\end{itemize}

\textbf{Physics and Mathematics Video Crrator on Bilibili} \hfill \textbf{Sep 2021 - Present}
\begin{itemize}[leftmargin=1.5em]
    \item Created and shared educational videos on physics and math topics, reaching a broad audience and achieving over 160,000 views on \href{https://www.bilibili.com/video/BV1th411W7xu/}{the most popular video}
\end{itemize}

\textbf{Organizer and Lecturer of Seminar of Physics at the University of Macau} \hfill \textbf{Feb 2022 - May 2022}
\begin{itemize}[leftmargin=1.5em]
    \item Initiated a lecture series with a colleague to bridge the curriculum gap at the University of Macau, providing in-depth instruction in advanced physics and mathematics to supplement formal education
    \item Compiled and distributed a guide for incoming physics and chemistry students to navigate academic and research paths
    \item Consistently delivered bi-weekly seminars and made the course content (\href{https://github.com/Chisen-Lupus/Seminar-of-Physics-UM/blob/main/SPUM%20102%20The%20tools%20of%20physical%20tool.pdf}{SPUM 102 The tools of physical tools}) publicly available on \href{https://www.youtube.com/watch?v=nQkv03r-XeQ&list=PLV9fHDZW7hHWQ9rrAk7c9kdeV-Lqyt7pV&index=10}{Youtube}, expanding educational outreach
\end{itemize}

\end{section}

\begin{section}{EXTRACURRICULAR EXPERIENCES}

\textbf{Director} of \textit{Physics Society, University of Macau (Macau SAR, China)} \hfill \textbf{Aug 2020 - Feb 2021} 
\begin{itemize}[leftmargin=1.5em]
    \item Established and expanded the Physics Society, leading promotional efforts and significantly growing its membership
    \item Guided undergraduates through the China Undergraduate Physics Tournament, enhancing the society's academic community
\end{itemize}

\textbf{Student Helper} at \textit{Department of Physics and Chemistry, University of Macau (Macau SAR, China)} \hfill \textbf{Jul 2020 - Oct 2020} 
\begin{itemize}[leftmargin=1.5em]
    \item Handled equipment procurement processes and managed budget recommendations, streamlining the department's operations
\end{itemize}

\textbf{Organizer} of \textit{Beijing Astronomy and Astrophysics Olympiad (Beijing \& Guangdong, China)} \hfill \textbf{Jan 2018 - Apr 2018} 
\begin{itemize}[leftmargin=1.5em]
    \item Managed the organization and execution of the 2018 Olympiad, securing participation from schools across China and facilitating a knowledge-rich event
\end{itemize}

\textbf{President} of \textit{Beijing Youth Astronomy Union (Beijing, China)} \hfill \textbf{Aug 2017 - Aug 2018} 
\begin{itemize}[leftmargin=1.5em]
    \item Directed an educational initiative for the National Astronomy Olympiad and engaged the public with astronomy through various outreach activities
    \item Managed the WeChat account ``北京市中学生天文联盟'', achieving widespread readership with posts exceeding 100,000 views
\end{itemize}

\end{section}

\end{document}


% 加talk conference