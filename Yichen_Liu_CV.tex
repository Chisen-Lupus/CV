\documentclass[11pt]{article}
\usepackage[slantfont, boldfont]{xeCJK}
\usepackage{amsmath}
\usepackage{enumitem}
\usepackage[left=0.4in,top=0.8in,right=0.4in,bottom=0.8in, letterpaper]{geometry}
\usepackage{hyperref}
\usepackage[x11names]{xcolor}
\hypersetup{
    colorlinks=true,
    linkcolor=blue,
    filecolor=blue,
    citecolor = black,      
    urlcolor=Blue2,
    }
\linespread{1 }%行距 1
% \setlength{\parskip}{-1.5ex}%段距 1.5ex
\setlength{\parindent}{0em}%缩进 0em
\usepackage{fancyhdr}
\pagestyle{fancy}
\fancyhf{}
\cfoot{\thepage\hspace{1pt} OF \pageref*{LastPage}}
\renewcommand\headrulewidth{0pt}
\setlength{\headheight}{0.6in}%加上之后其余页就同步了
\setlength{\footskip}{5ex}%加上之后其余页就同步了
\setlength{\headsep}{2ex}
\usepackage{titlesec}
\titleformat{\section}{\vspace{-1.5ex}\bf}{}{0em}{}[\hrule\vspace{-1.5ex}]
\usepackage{lastpage}
\usepackage{fontspec}
% \setmainfont{Garamond}
% \setmainfont{Cambria}
% \setmainfont{Georgia}%mac ok
% \setmainfont{Georgia}
\setmainfont{Book Antiqua}
\usepackage{enumitem}
\setenumerate[1]{itemsep=0pt,partopsep=0pt,parsep=\parskip,topsep=5pt}
\setitemize[1]{itemsep=0ex,partopsep=0ex,parsep=0ex,topsep=0ex}
\setitemize[2]{itemsep=0ex,partopsep=0ex,parsep=0ex,topsep=0ex}
% \setdescription{itemsep=0pt,partopsep=0pt,parsep=\parskip,topsep=5pt}
\usepackage{academicons}
\newcommand{\orcid}[1]{\href{https://orcid.org/#1}{\textcolor[HTML]{A6CE39}{\aiOrcid}}}
\newcommand{\googlescholar}[1]{\href{https://scholar.google.com.hk/citations?user=#1}{\textcolor[HTML]{3983FE}{\aiGoogleScholar}}}
\usepackage{fontawesome}
\newcommand{\github}[1]{\href{https://github.com/#1}{\textcolor[HTML]{000000}{\faGithub}}}
\usepackage{booktabs}% http://ctan.org/pkg/booktabs
\newcommand{\tabitem}{~~\llap{\textbullet}~~}

\chead{
    {\bf\underline{\large{YICHEN LIU}}}\\
    \vspace{1ex}
    tel: +1-(447) 902-2638 | e-mail: \href{mailto:yl127@illinois.edu}{yl127@illinois.edu} | web: \href{https://cheysen.fit}{https:$/\!/$cheysen.fit} | \orcid{0000-0003-4247-0169} \github{Chisen-Lupus} \googlescholar{GRjhRLUAAAAJ}
}

\begin{document}

%\setlength{\headsep}{0pt}

\begin{section}{EDUCATION}

\textbf{University of Illinois at Urbana-Champaign} (Illinois, U.S.) | College of Liberal Arts \& Sciences \hfill \textbf{Aug 2022 - Present}
\begin{itemize}[leftmargin=1.5em]
    \item Bachelor of Science - \textbf{Astrophysics} and \textbf{Mathematics}\hfill \textbf{4.00/4.00} 
    \item Minor in \textbf{Physics}, \textbf{Computer Science}, and \textbf{Chemistry}
    \item Enrolled in the \textbf{LAS Honors program} 
\end{itemize}

\textbf{University of Macau} (Macau SAR, China) | Faculty of Science and Technology \hfill \textbf{Aug 2019 - May 2022}
\begin{itemize}[leftmargin=1.5em]
    \item Major in \textbf{Applied Physics and Chemistry} \hfill \textbf{3.40/4.00} (3.81 in the third year)
    \item Enrolled in the \textbf{Honours College}
    % \item Director of \textbf{University of Macau Physics Society} \hfill Aug 2020 - Feb 2021
\end{itemize}

\textbf{Summer sessions}
\begin{itemize}[leftmargin=1.5em]
    \item \textbf{University of California, Berkeley} (Online) \hfill 4.000/4.000  \textbf{Jun 2022 - Aug 2022} 
    \item \textbf{Shanghai Jiao Tong University} (Shanghai, China) \hfill 4.00/4.00  \textbf{Jun 2021 - Jul 2021}
\end{itemize}
    
\end{section}

\begin{section}{RSEARCH EXPERIENCES}

\textbf{Department of Astronomy} (Illinois, U.S.) - \textit{research assistant} \hfill \textbf{Sep 2022 - Present}
\begin{itemize}[leftmargin=1.5em]
    \item Supervisor - \href{mailto:xinliuxl@illinois.edu}{Professor Xin Liu}
    \item Topic 1 - galaxy deblending using machine learning \hfill \textbf{Sep 2022 - Present}
    \begin{itemize}[leftmargin=1.5em]
        \item Examined the source extraction pipelines of \href{https://github.com/burke86/astrodet}{\texttt{Astrodet}} and \href{https://github.com/burke86/astro_rcnn}{\texttt{Astro R-CNN}} using \href{https://github.com/kbarbary/sep/tree/v1.1.x}{\texttt{Sep}} and \href{https://github.com/pmelchior/scarlet}{\texttt{Scarlet}}
        \item Arranged test runs of different models and configurations based on PhoSim data on Hardware - Accelerated Learning (HAL) cluster using \href{https://github.com/facebookresearch/detectron2}{\texttt{Detectron2}}
        \item Wrote code to introduce Transformer models such as MViT and VitDet into the pipeline
        \item Enrolled in the NCSA SPIN internship in Summer 2023
    \end{itemize}
    \item Topic 2 - DES SED fitting \hfill Feb 2023 - Present
    \begin{itemize}[leftmargin=1.5em]
        \item Generated SED spectra of the sources in the DES and WISE catalogs using \href{https://cigale.lam.fr/}{\texttt{CIGALE}} 
        \item Compared and labelled potential AGNs in the catalog
    \end{itemize}
    % \item Topic 3 - \color{red}{[SPIN intern]} \hfill {\color{red}[\textbf{Jun 2023 - Aug 2023}]}
    % \begin{itemize}[leftmargin=1.5em]
    %     \item {\color{red}[details]}
    % \end{itemize}
\end{itemize}

\textbf{National Observatory of China} (Beijing, China) - \textit{summer research internship} \hfill \textbf{Jun 2022 - Aug 2022}
\begin{itemize}[leftmargin=1.5em]
    \item Supervisor - \href{mailto:chjwu@bao.ac.cn}{Professor Chaojian Wu}
    \item Topic - meteor slitless spectrum
    \begin{itemize}[leftmargin=1.5em]
        \item Computed the spectrum of Gemini meteors captured by DSLR
        \item Analyzed the intensitities of Sodium and Magnesium using Python
        \item Wrote code to use machine learning to identify meteors from the meteors and do photometry automatically
    \end{itemize}
\end{itemize}

\textbf{Institute of Applied Physics \& Materials Engineering} (Macau SAR, China) - \textit{research assistant} \hfill \textbf{Aug 2019 - May 2022}
\begin{itemize}[leftmargin=1.5em]
    \item Supervisor - \href{mailto:hcliu@um.edu.mo}{Professor Hongchao Liu}
    \item Topic 1 - ghost imaging under complex environment  \hfill Aug 2019 - Aug 2021
    \begin{itemize}[leftmargin=1.5em]
        \item Studied the concepts of single-pixel imaging \& ghost imaging
        \item Managed ghost imaging experiments in MATLAB (measure image quality based on different equipment and algorithms), analyze the data, and author reports for project supervisors
        \item Reviewed latest studies on ghost imaging and presented research summaries at staff meetings
    \end{itemize}
    \item Topic 2 - anti-loss image encryption based on ghost imaging \hfill Sep 2021 - Feb 2022
    \begin{itemize}[leftmargin=1.5em]
        \item Conducted experiments on ghost imaging using the optimized patterns 
        \item Designed Python algorithms based on the concept of compressive sensing 
        \item Conducted the computational simulations of ghost imaging and integrated the results
        \item Authored research paper, becoming \href{https://www.tdm.com.mo/en/news-detail/683438?isvideo=false&lang=en&category=all}{the first undergraduate student in the department publishing research paper} 
    \end{itemize}
    \item Topic 3 - ghost imaging using recurrent neural network  \hfill Mar 2022 - May 2022
    \begin{itemize}[leftmargin=1.5em]
        \item Operated lasers and other equipments in collaboration with postgraduate students
        \item Summarized and verified the existing ghost imaging methods using nerural networks
        \item Designed Python algorithms based on recurrent and convolutional neural network for ghost imaging 
    \end{itemize}
\end{itemize}

\end{section}

\newpage

\begin{section}{PUBLICATIONS}
    
\begin{enumerate}[leftmargin=1.5em]
    \item  \textbf{Yichen Liu}, Peixia Zheng, and Hong-Chao Liu, \href{https://opg.optica.org/oe/fulltext.cfm?uri=oe-30-9-14073&id=471300}{``Anti-loss-compression image encryption based on computational ghost imaging using discrete cosine transform and orthogonal patterns,"} Opt. Express 30, 14073-14087 (2022)
    \item Peixia Zheng, \textbf{Yichen Liu}, and Hong-Chao Liu, \href{http://www.irla.cn/cn/article/doi/10.3788/IRLA20211058}{``Single-pixel imaging and metasurface imaging (Invited),"} Infrared and Laser Engineering (红外与激光工程) 50.12, 20211058-1 (2022)
\end{enumerate}

\end{section}

% \newpage

\begin{section}{OBSERVATION EXPERIENCE}
    
    \begin{itemize}[leftmargin=1.5em]
        \item Cerro Tololo Inter-American Observatory, Blanco 4m / DECam: 3 nights observation \hfill Jan 2023 - Apr 2023
    \end{itemize}

\end{section}

\begin{section}{AWARDS}

    \begin{itemize}[leftmargin=1.5em]
        \item University of Illinois Dean's Honour List \hfill Jan 2023
        \item Smart Star Sponsorship for studies at University of California, Berkeley \hfill Jun 2022
        \item University of Macau Dean's Honour List (2020 first semester and 2022) \hfill Aug 2022
        \item Residential College Summer Programme Sponsorship for studies at Shanghai Jiao Tong university \hfill May 2021
        \item Third Prize, China Undergraduate Physics Tournament \hfill Oct 2020
        \item National Team Leader at the 2019 European Union Contest for Young Scientists \hfill Sep 2019
        \item University of Macau Full Scholarship (tuition \& accommodation, 2019-2021) \hfill Aug 2019
        \item Bronze Medal, International Olympiad of Astronomy and Astrophysics \hfill Nov 2018
        \item First Prize, China Adolescents' Science and Technology Innovation Contest \hfill Aug 2018
        % \item Second Prize, Deng Feng National Contest on Science and Innovation \hfill Aug 2018
        \item Second Prize, China National Astronomy Olympiad \hfill May 2018
    \end{itemize}
    
\end{section}

\begin{section}{OUTREACH}

\textbf{Youtuber in physics and mathematics} (Online) \hfill {Sep 2021 - Present}
\begin{itemize}[leftmargin=1.5em]
    \item Occasionally live stream or publish videos in Bilibili platform, offering public education resources in Chinese
    \item Topics include mathematics and physics that I self-learned
    \item \href{https://www.bilibili.com/video/BV1th411W7xu/}{The most popular video} obtained more than 160,000 watchings
\end{itemize}

\textbf{Seminar of Physics at the University of Macau} (Macau SAR, China) \hfill {Feb 2022 - May 2022}
\begin{itemize}[leftmargin=1.5em]
    \item This is a series of unofficial lectures organized by me and my classmate, \href{http://runawayfancy.me/}{Jiheng Duan}, willing to offer the contents that the University of Macau's curriculum does not provide, such as classical mechanics, supplementing the theoretical basis of  the future research and studies in physics for the DPC students
    \item Conducted and arranged the class meetings twice a week
    \item Prepared and taught the lecture \textit{SPUM 102 The tools of physical tools}, including complex variables, $\Gamma$ functions, integral transforms, $\delta$ functions, and Green functions \hfill {\footnotesize \href{https://github.com/Chisen-Lupus/Seminar-of-Physics-UM/blob/main/SPUM%20102%20The%20tools%20of%20physical%20tool.pdf}{|Syllabus}}
    \item The recordings of the lectures are available on \href{https://www.youtube.com/watch?v=nQkv03r-XeQ&list=PLV9fHDZW7hHWQ9rrAk7c9kdeV-Lqyt7pV&index=10}{Youtube}
    % \item Wrote the academic guideline for the freshmans 
\end{itemize}

\end{section}
    
\begin{section}{ACTIVITIES}

\textbf{Department of Astronomy} (Illinois, U.S.) - \textit{undergraduate tutor} \hfill \textbf{Jan 2023 - Present}
\begin{itemize}[leftmargin=1.5em]
    \item Planned and facilitated collaborative tutoring sessions for Astronomy-program major students enrolled in targeted core major courses
\end{itemize}

\textbf{Personal 25-centimeter Remote Observatory} (Hebei, China) - \textit{astrophotographer} \hfill \textbf{Jan 2018 - Jul 2022}
\begin{itemize}[leftmargin=1.5em]
    \item Identified a suitable site in China and built a 2$\times$2-meter storage facility with internet access and a retractable roof
    \item Selected, assembled, and tested the equipment; successfully developed a remotely-operated facility
    \item \href{https://cheysen.fit/astrophotography/}{Regularly captured emission nebulae}
\end{itemize}

\textbf{University of Macau Physics Society} (Macau SAR, China) - \textit{director} \hfill \textbf{Aug 2020 - Feb 2021}
\begin{itemize}[leftmargin=1.5em]
    \item Founded the University's physics society and promoted its activities on social media platforms
    \item Significantly expanded the Society's membership through effective outreach activities and university club fairs
    \item Organized and led a team of undergraduate students at the 2020 China Undergraduate Physics Tournament% and won Third Prize
\end{itemize}

\newpage

\textbf{Department of Physics and Chemistry} (Macau SAR, China) - \textit{student helper} \hfill \textbf{Jul 2020 - Oct 2020}
\begin{itemize}[leftmargin=1.5em]
    \item Procured equipment for physics-related research (requested product quotes and negotiated contracts with suppliers) 
    \item Tracked the department's procurement budget and developed budgeting recommendations
    \item Purchased supplies for the 2020 China Undergraduate Physics Tournament
\end{itemize}

\end{section}

\begin{section}{TECHNICAL SKILLS}

\begin{tabular}{rl}
    Skilled in: & Python, \LaTeX, MATLAB, Git, and Shell Bash/Zsh \\
    Basic Knowledge: & C/C++, Mathematica, Julia, docker, and Java \\
    Softwares: & MaxIm DL, COMSOL, Altium Designer, KiCAD, Solidworks, Cinema 4D, and SPSS \\
    Often-used Packages: & \href{https://www.astropy.org/}{\texttt{AstroPy}}, \href{https://github.com/pmelchior/scarlet}{\texttt{Scarlet}}, \href{https://pytorch.org/}{\texttt{PyTorch}}, \href{https://github.com/facebookresearch/detectron2}{\texttt{Detectron2}}, and \href{https://cigale.lam.fr/}{\texttt{CIGALE}} \\
    Contributions: & \href{https://github.com/burke86/astrodet}{\texttt{Astrodet}}: instance segmentation of galaxies and stars \\
    & \href{https://github.com/Chisen-Lupus/metspec}{\texttt{Metspec}}: Auto-detection and photometry of meteor slitless spectrum \\
    & \href{https://github.com/Chisen-Lupus/DES-SED-fitting}{\texttt{DES-SED-Fitting}}: SED fitting and classification of DES sources\\
    & \href{https://github.com/gnarayan/decat_pointings}{\texttt{DECat-pointings}}: working repository of DECam
\end{tabular}

\end{section} 

\end{document}